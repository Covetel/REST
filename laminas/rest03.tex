% Copyleft 2010 by Walter Vargas <walter@covetel.com.ve>

\documentclass[9pt]{beamer}
\usepackage{listings}
\lstset{
    language=HTML 
}
\lstset{
    basicstyle=\scriptsize, 
    stringstyle=\ttfamily,
    showstringspaces=false, 
    numbers=left,
    numberstyle=\scriptsize,
    tabsize=2,
}

% Configuración de la apariencia. 

\usetheme{Szeged}
\usecolortheme{beaver}
\usefonttheme[onlylarge]{structurebold}
\setbeamerfont*{frametitle}{size=\normalsize,series=\bfseries}

\usepackage{color}
\definecolor{lightgray}{rgb}{.9, .9, .9}
\definecolor{darkgray}{rgb}{.4, .4, .4}
\definecolor{purple}{rgb}{0.65,  0.12,  0.82}
\definecolor{azulito}{HTML}{0066CC}

\lstdefinelanguage{HTML}{
keywords={xml, DOCTYPE, head,  body, html}, 
keywordstyle=\color{azulito}\bfseries, 
ndkeywords={title, option, form, textarea, table, th, tr, td, p, br, input,
button, frame, iframe, div, optgroup, select, fieldset,  legend, h1, h2, h3,
h4, h5, h6, p, blockquote, address, em,strong, abbr, acronym, cite, dfn, code,
kbd, samp, var, q, ul, li, dl, dt, dd}, 
ndkeywordstyle=\color{blue}\bfseries, 
identifierstyle=\color{black}, 
sensitive=false, 
comment=[l]{//}, 
morecomment=[s]{/*}{*/}, 
stringstyle=\color{azulito}\ttfamily, 
morestring=[b]', 
morestring=[b]"
}

\lstset{
language=HTML, 
backgroundcolor=\color{lightgray}, 
extendedchars=true, 
basicstyle=\scriptsize\ttfamily, 
showstringspaces=false, 
showspaces=false, 
numbers=left, 
numbersstyle=\footnotesize, 
numbersep=9pt, 
tabsize=2, 
breaklines=true, 
showtabs=false, 
captionpos=b 
}


% Paquetes
\usepackage[utf8]{inputenc}
\usepackage[spanish]{babel}


% Titulo
\title[REST] {WebService REST con Catalyst}
\author[Walter Vargas]{ info@covetel.com.ve \inst{1}}
\subtitle{Catalyst::Controller::REST}
\institute[covetel.com.ve]{ \inst{1} Cooperativa Venezolana de Tecnologías Libres R.S. }
\date

\begin{document}

\section{Catalyst::Controller::REST}

\begin{frame}{Catalyst::Controller::REST} % (fold)
    Catalyst::Controller::REST implementa un mecanismo para crear servicios
    RESTful en Catalyst. Extiende el mecanismo convencional de despacho de Catalyst
    para soportar una serie de subrutinas que son invocadas en base al verbo
    HTTP de la solicitud. \\[0.5cm]

    \textit{Catalyst::Controller::REST} maneja de manera transparente todo el proceso de
    serialización y deserialización por nosotros. 
\end{frame}

\subsection{Synopsis} % (fold)


\begin{frame}[fragile]{Synopsis} % (fold)
\begin{lstlisting}
package TestServer::Controller::REST;
use Moose;
use namespace::autoclean;

BEGIN {extends 'Catalyst::Controller::REST'; }

sub prueba : Local : ActionClass('REST') {}

sub prueba_GET {
    my ($self, $c) = @_;
    # Retornar 200 OK, con los datos en entity serializados en el body
    $self->status_ok(
        $c,
        entity => {
            nombre   => 'elnombre',
            apellido => 'elapellido',
        }
    );
}    
\end{lstlisting}
\end{frame}

% subsection Synopsis (end)

\subsection{Pasos para crear una controladora REST} % (fold)


\begin{frame}[fragile, allowframebreaks]{Una controladora REST en pocos pasos} % (fold)

    \begin{enumerate}
        \item Crear una controladora de ejemplo \texttt{REST}
        \begin{verbatim}
$ script/testserver_create.pl controller REST
        \end{verbatim}
        \item Editamos el archivo \texttt{lib/TestServer/Controller/REST.pm}
        \item La controladora debe extender de
        \textit{Catalyst::Controller::REST} 
        \begin{lstlisting}
BEGIN {extends 'Catalyst::Controller::REST'; }
        \end{lstlisting}
        \item Creamos una acción base utilizando la clase de acción
        \texttt{REST}
        \begin{lstlisting}
sub prueba : Local : ActionClass('REST') {}
        \end{lstlisting}

        \item Ahora debemos crear una subrutina por cada verbo HTTP que
        quisieramos utilizar. 
        \begin{lstlisting}
sub prueba_GET {
    my ($self, $c) = @_;
    # Retornar 200 OK, con los datos en entity serializados en el body
    $self->status_ok(
        $c,
        entity => {
            nombre   => 'elnombre',
            apellido => 'elapellido',
        }
    );
}
        \end{lstlisting}

        \item Iniciamos el servidor de desarrollo de \textit{Catalyst}. 
        \begin{verbatim}
$ script/testserver_server.pl -r
        \end{verbatim}

        \footnotesize{
        \item Probamos el verbo GET con \texttt{curl}
        \begin{verbatim}
$ curl -X GET -i -H "Content-Type: text/html" localhost:3000/rest/prueba
        \end{verbatim}
        }
    \end{enumerate}

    \textbf{Resumen:} Todas las peticiones GET hacia la ruta
    \texttt{/rest/prueba} se procesan con la subrutina \texttt{prueba\_GET} y
    así sucesivamente para cada uno de los verbos HTTP.
\end{frame}

\begin{frame}[fragile]{Métodos no implementados} % (fold)
    Cualquier petición que utilice un método HTTP no implementado sera
    respondida con el mensaje \textit{405 Method Not Allowed} y en la cabecera se
    especificarán los métodos soportados. \\[0.2cm]

    \textbf{Ejemplo:} 
   \footnotesize{
    \begin{verbatim}
$ curl -X PUT -i -H "Content-Type: text/html" localhost:3000/rest/prueba 

HTTP/1.0 405 Method Not Allowed
Connection: close
Date: Tue, 27 Sep 2011 16:00:41 GMT
Allow: GET
Allow: POST
Content-Length: 64
Content-Type: text/plain
Status: 405
X-Catalyst: 5.80032

Method PUT not implemented for http://localhost:3000/rest/prueba
    \end{verbatim}
    }

\end{frame}

\begin{frame}{\texttt{OPTIONS}} % (fold)
    Si no se especifica un \texttt{handler}, la controladora va a responder a
    cualquier petición \texttt{OPTIONS} con un mensaje \texttt{200 OK},
    agregando los verbos permitidos en la cabecera de la respuesta de manera
    automática.
\end{frame}

% subsection Pasos para crear una controladora REST (end)

\subsection{Serialización} % (fold)

\begin{frame}[fragile]{Serialización} % (fold)
    \texttt{Catalyst::Controller::REST} va automáticamente a serializar
    nuestras respuestas, y a deserializar cualquier petición \texttt{POST},
    \texttt{PUT}, \texttt{OPTIONS}. Evalua cual serializador debe utilizar, mapeando
    el tipo de contenido (\texttt{content-type}) a un módulo de serialización.\\[0.5cm]

    El tipo de contenido (\texttt{content-type}) se selecciona en base a: 

\footnotesize{

    \begin{itemize}
        \item \textbf{Content-Type Header}, si la petición viene con un
        atributo \texttt{Content-Type} definido,  se utilizara este valor. 
\begin{verbatim}
$ curl -X GET -i -H "Content-Type: application/json" localhost:3000/rest/prueba
\end{verbatim}

        \item \textbf{content-type query parameter},  Si la petición es GET,
        puedes suministrar el parámetro content-type junto con la lista de
        parámetros.
\begin{verbatim}
$ curl -X GET -i  localhost:3000/rest/prueba -d "content-type=text/xml"
\end{verbatim}

        \item \textbf{Accept Header}, finalmente si el cliente provee un
        \textit{Accept Header}, se evalua y se utiliza la mejor opción
        disponible para serializar. 
\begin{verbatim}
$ curl -X GET -i -H "Accept: application/json" localhost:3000/rest/prueba
\end{verbatim}
    \end{itemize}

    }
\end{frame}

\subsubsection{Serializadores Disponibles} 

\begin{frame}[fragile, allowframebreaks]{Serializadores Disponibles} % (fold)

Un mecanismo de serialización dado sólo esta disponible si tiene instalado los
módulos subyacentes. Por ejemplo, no puedes utilizar \texttt{XML::Simple} si el
módulo no está disponible. \\[0.2cm]

Adicionalmente, cada serializador tiene sus peculiaridades en cuanto a que
estructuras de datos serán manejadas correctamente.\\

En la siguiente lista podrá ver los mecanismos de serialización más comunes
disponibles.

\begin{itemize}
    \item \texttt{text/x-yaml} \verb|=>| \texttt{YAML::Syck} \\
    Devuelve un YAML generado por el módulo \href{http://search.cpan.org/perldoc?YAML%3A%3ASyck}{YAML::Syck}

    \item  \texttt{text/html} \verb|=>| \texttt{YAML::HTML} \\
    Utiliza los módulos
    \href{http://search.cpan.org/perldoc?YAML%3A%3ASyck}{YAML::Syck} y
    \href{http://search.cpan.org/perldoc?URI%3A%3AFind}{URI::Find} para generar
    YAML con todas las URLs en forma de \texttt{hyperlinks}.


    \item \texttt{application/json} \verb|=>| \texttt{JSON} \\
    Utiliza el módulo \href{http://search.cpan.org/perldoc?JSON}{JSON} para
    generar la salida en formato JSON. Se recomienda fuertemente tener el
    módulo \texttt{JSON::XS} instalado. El tipo de contenido
    \texttt{text/x-json} es soportado pero esta \textit{deprecated} y al
    utilizarlo va a recibir mensajes de advertencia en los logs.

    \item \texttt{x-data-dumper} \verb|=>|  \texttt{Data::Serializer}\\
    Utiliza el módulo \texttt{Data::Serializer} para generar una salida en el
    formato \texttt{Data::Dumper}

    \item \texttt{application/x-storable} \verb|=>| \texttt{Data::Serializer}\\
    Utiliza el módulo \texttt{Data::Serializer} para generar una salida en el
    formato \href{http://search.cpan.org/perldoc?Storable}{Storable} de Perl.

    \item \texttt{text/x-php-serialization} \verb|=>| \texttt{Data::Serializer}\\
    Utiliza el módulo \texttt{Data::Serializer} para generar una salida en el
    formato \texttt{PHP::Serialization}

    \item \texttt{text/xml} \verb|=>| \texttt{XML::Simple} \\

    Utiliza \texttt{XML::Simple} para generar la salida en formato XML. Este
    método no es recomendado para trabajo pesado con XML. 


\end{itemize}

\end{frame}
% subsection Serialización (end)

\subsection{Status Helpers} % (fold)

\begin{frame}{Status Helpers} % (fold)
    Dado que gran parte de REST es HTTP, fueron creados los \textit{helpers} de
    estado. Utilizar estos \textit{helpers} garantiza que la respuesta esta
    correctamente formada conforme a los códigos, cabeceras y entidades HTTP
    1.1. \\[0.5cm]

    Estos \textit{helpers} cumplen con la especificación HTTP 1.1. Están
    implementados como subrutinas convencionales, por lo cual es necesario
    pasar como primer argumento la variable contexto de Catalyst (\texttt{\$c}).

\end{frame}

\begin{frame}[fragile]{\texttt{status\_ok}} % (fold)
    Devuelve como respuesta la cadena \textit{200 OK}. Toma una entidad
    (\texttt{entity}) y la serializa.\\[0.5cm]

    \textbf{Ejemplo}
    \begin{lstlisting}
$self->status_ok( $c, entity => { radiohead => "Is a good band!" } );
    \end{lstlisting}
    
\end{frame}

\begin{frame}[fragile]{\texttt{status\_created}} % (fold)
    Devuelve como respuesta la cadena \textit{201 CREATED}. Toma una entidad
    (\texttt{entity}) y la serializa, adicionalmente se debe definir un
    atributo \texttt{location} que contiene la ruta para alcanzar al recurso
    reicentemente creado. \\[0.5cm]

    \textbf{Ejemplo}
    \begin{lstlisting}
$self->status_created(
   $c,
   location => $c->req->uri->as_string,
   entity => {
       radiohead => "Is a good band!",
   }
 );
    \end{lstlisting}

\end{frame}

\begin{frame}[fragile]{\texttt{status\_no\_content}} % (fold)
    Devuelve como respuesta la cadena \textit{204 NO CONTENT}. Se utiliza en
    los casos en que la petición fue correctamente procesada, pero no hace
    falta devolver información al agente de usuario. \\[0.5cm]

    Una respuesta 204 no debe incluir cuerpo de mensaje, y siempre termina con
    la primera línea vacía despues de la cabecera.
\end{frame}

\begin{frame}[fragile]{\texttt{status\_multiple\_choices}} % (fold)
    Devuelve como respuesta la cadena \textit{300 MULTIPLE CHOICES}. Toma una
    entidad (\texttt{entity}) para serializarla, la cual debería proporcionar
    una lista de posibles rutas para alcanzar los recursos. También puede
    contener un parámetro opcional \texttt{location} para definir una ruta de
    preferencia. \\[0.5cm]

    Se utiliza en los casos en que el recurso solicitado se corresponde con
    algundo de entre una lista de recursos alternativos. Ha de ser el propio
    agente de usuario (navegador, normalmente) quien elija entre los diferentes
    recursos.  
\end{frame}

\begin{frame}[fragile]{\texttt{status\_bad\_request}} % (fold)
    Devuelve como respuesta la cadena \textit{400 BAD REQUEST}. Recibe como
    argumento un hash \texttt{messaje} que será convertida en el valor de \texttt{error} en
    la respuesta serializada. \\[0.5cm]

    \textbf{Ejemplo:}
    \begin{lstlisting}
$self->status_bad_request(
    $c,
    message => "No puedo darle lo que me pide",
  );
    \end{lstlisting}
\end{frame}

\begin{frame}[fragile]{\texttt{status\_not\_found}} % (fold)
    Devuelve como respuesta la cadena \textit{404 NOT FOUND}. Toma un hash
    \texttt{messaje} que será convertido en el valor de  \texttt{error} en la
    respuesta serializada. \\[0.5cm]

    \textbf{Ejemplo:} 

    \begin{lstlisting}
$self->status_not_found(
    $c,
    message => "No puede encontrar lo que estas buscando",
  );

    \end{lstlisting}

\end{frame}

\begin{frame}[fragile]{\texttt{status\_gone}} % (fold)
    Devuelve una cadena \texttt{410 GONE} como respuesta. Toma un hash
    \texttt{messaje} que será convertido en el valor de \texttt{error} en la
    respuesta serializada. \\[0.5cm]

    \texttt{41O GONE}: El recurso ya no está disponible, y no se espera que lo
    vuelva a estar en el futuro. \\[0.5cm]

    \textbf{Ejemplo: }

    \begin{lstlisting}
$self->status_gone(
    $c,
    message => "El documento fue eliminado.",
  );        
    \end{lstlisting}
\end{frame}

% subsection Status Helpers (end)



\end{document}
